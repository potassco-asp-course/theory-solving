% ----------------------------------------------------------------------
\newcommand{\scope}[1]{\ensuremath{\mathit{S}(#1)}}
\newcommand{\range}[1]{\ensuremath{\mathit{R}(#1)}}
\newcommand{\project}[2]{\ensuremath{{#1}|_{#2}}}
\newcommand{\abody}[1]{\project{\mathit{body}(#1)}{\mathcal{A}}} % {\ensuremath{\mathit{abody}(#1)}}
\newcommand{\cbody}[1]{\project{\mathit{body}(#1)}{\mathcal{C}}} % {\ensuremath{\mathit{cbody}(#1)}}
\newcommand{\catom}[1]{\project{\atom{#1}}{\mathcal{C}}} % {\ensuremath{\mathit{catom}(#1)}}
\newcommand{\aatom}[1]{\project{\atom{#1}}{\mathcal{A}}}
% ----------------------------------------------------------------------
\begin{frame}{Constraint Satisfaction Problem}
  \bigskip
  \begin{itemize}
  \item<1-> A \alert{constraint satisfaction problem} (CSP) consists of
    \begin{itemize}
    \item a set $V$ of variables,
    \item a set $D$ of domains, and
    \item a set $C$ of constraints
    \end{itemize}
  \item<2-> [] such that
    \begin{itemize}
    \item each \alert{variable} $v\in V$ has an associated \alert{domain} $\mathit{dom}(v)\in D$\/;
    \item a \alert{constraint} $c$ is a pair
      \(
      (S,R)
      \)
      consisting of a $k$-ary \alert{relation} $R$
      on a vector $S\subseteq V^k$ of variables,
      called the \alert{scope} of $R$
    \end{itemize}
    \medskip
  \item<3-> \structure{Note}
    For $S=(v_1,\dots,v_k)$,
    we have
    \(
    R\subseteq\mathit{dom}(v_1)\times\dots\times\mathit{dom}(v_k)
    \)
  \end{itemize}
\end{frame}
% ----------------------------------------------------------------------
\begin{frame}{Example}
  \begin{columns}
    \begin{column}{.35\textwidth}
      \begin{tabular}[t]{ccccc}
         &s&e&n&d\\
        +&m&o&r&e\\
        \hline
        m&o&n&e&y
      \end{tabular}
    \end{column}
    \begin{column}{.35\textwidth}
      Each letter corresponds exactly to one digit and all variables have to be
      pairwisely distinct
    \end{column}
  \end{columns}
\pause
\bigskip\bigskip
\alt<3>
{% - alternative layer ---

  \begin{columns}
    \begin{column}{.35\textwidth}
      \begin{tabular}[t]{ccccc}
         &9&5&6&7\\
        +&1&0&8&5\\
        \hline
        1&0&6&5&2
      \end{tabular}
    \end{column}
    \begin{column}{.35\textwidth}
      The example has exactly one solution
    \end{column}
  \end{columns}
\[
\{\ s\mapsto9, e\mapsto5, n\mapsto6, d\mapsto7, m\mapsto1, o\mapsto0, r\mapsto8, y\mapsto2\ \}
\]
}{% - first layer ---
\begin{tabbing}
 \qquad
 \=$V =\;$\=$\{s,e,n,d,m,o,r,y\}$\\[5pt]
 \>$D =$\>$\{\domain{v}= \{0,\dots,9\}\mid v \in V\}$\\[5pt]
 \>$C =$\>$\{\,$\=$(\,$\=$\vec{V},$~~~\=$allDistinct(V)\,)$,\\
 \>\>\>$($\>$\vec{V},$\>$s\times1000+e\times100+n\times10+d +$\\
 \>\>\>\>\>$m\times1000+o\times100+r\times10+e ==$\\
 \>\>\>\>\>$m\times10000+o\times1000+n\times100+e\times10+y),$\\
 \>\>\>$($\>$(m),$\>$ m == 1)\,\}$\\
\end{tabbing}}
\end{frame}
% ----------------------------------------------------------------------
\begin{frame}{Constraint satisfaction problem}
  \bigskip
  \begin{itemize}
  \item<1-> \structure{Notation}
    We use $\scope{c}=S$ and $\range{c}=R$ to access the scope and the relation
    of a constraint $c=(S,R)$
    \medskip
  \item<2-> For an assignment
    \(
    A: V\to \bigcup_{v\in V}\domain{v}
    \)
    and a constraint $(S,R)$
    with scope
    \(
    S=(v_1,\dots,v_k)
    \),
    define
    \[
    \mathit{sat}_C(A)
    =
    \{
    c\in C \mid A(\scope{c}) \in \range{c}
    \}
    \]
    where
    \(
    A(S)
    =
    (A(v_1),\dots,A(v_k))
    \)
  \end{itemize}
\end{frame}
% ----------------------------------------------------------------------
\begin{frame}{Constraint Answer Set Programming}
  \bigskip
  \begin{itemize}
  \item<1-> A \alert{constraint logic program}~$P$ is a logic program over an extended
    alphabet $\mathcal{A}\cup\mathcal{C}$
    where
    \begin{itemize}
    \item $\mathcal{A}$ is a set of \alert{regular atoms} and
    \item $\mathcal{C}$ is a set of \alert{constraint atoms},
    \end{itemize}
    such that $\head{r}\in\mathcal{A}$ for each $r\in P$
    \medskip
  \item<2-> Given a set of literals~$B$ and some set~$\mathcal{B}$ of atoms,
    we define
    \\\(\quad
    \project{B}{\mathcal{B}}
    =
    (\poslits{B}\cap\mathcal{B})
    \cup
    \{\neg a\mid a\in\neglits{B}\cap\mathcal{B}\}
    \)
  \end{itemize}
\end{frame}
% ----------------------------------------------------------------------
\begin{frame}{Constraint Answer Set Programming}
  \bigskip
  \begin{itemize}
  \item<1-> We identify constraint atoms with constraints via a function
    \[
    \gamma: \mathcal{C}\to C
    \]
  \item<1-> Furthermore,
    \(
    \gamma(Y)=\{\gamma(c)\mid c\in Y\}
    \)
    for any $Y\subseteq\mathcal{C}$
    \medskip
  \item<2-> \structure{Note}
    Unlike regular atoms $\mathcal{A}$, constraint atoms~$\mathcal{C}$
    are not subject to the unique names assumption, eg.\
    \begin{itemize}
    \item[] $\gamma(x<y) \ = \ \gamma(((-y-1)\leq -(x+1))\wedge (x\neq y))$
    \end{itemize}
    \medskip
  \item<3-> A constraint logic program $P$ is associated with a CSP\\
    as follows
    \begin{itemize}
    \item
      \(
      C[P]=\gamma(\atom{P}\cap\mathcal{C}) % (\catom{P})
      \),
    \item $V[P]$ is obtained from the constraint scopes in $C[P]$,
    \item $D[P]$ is provided by a declaration
    \end{itemize}
  \end{itemize}
\end{frame}
% ----------------------------------------------------------------------
\begin{frame}{Constraint Answer Set Programming}
  \begin{itemize}
  \item<1-> Let~$P$ be a constraint logic program over $\mathcal{A}\cup\mathcal{C}$ and \\
    let $A: V[P]\to D[P]$ be an assignment,

  \item<1-> [] define the \alert{constraint reduct} of as $P$ wrt $A$ as follows
    \[
    \begin{array}{rcl}
        P^A
        &=&
        \{\
        \head{r}\leftarrow\abody{r}
        \mid r\in P,\\
      &&\qquad\qquad
      \gamma(\poslits{\cbody{r}})
      \subseteq\mathit{sat}_{C[P]}(A)
      ,\\
      &&\qquad\qquad
      \gamma(\neglits{\cbody{r}})
      \,\cap\,
      \mathit{sat}_{C[P]}(A)
      =
      \emptyset
      \ \}
    \end{array}
    \]
  \item<2-> A set $\alert<3>{X}\subseteq\mathcal{A}$ of (regular) atoms
    is a \alert{constraint answer set} of~$P$ \alert<3>{wrt~$A$}, if
    $X$ is an stable model of~$P^A$.\pause%\pause
  \item<3->  \structure{Note}
    That is, if $X$ is the $\subseteq$-smallest model of $(P^A)^X$
  \end{itemize}
\end{frame}
% ----------------------------------------------------------------------
\begin{frame}{Some Constraint Answer Set Programming (CASP) systems}
  \bigskip
  \begin{itemize}
  \item<1-> \alert{\adsolver}
    \begin{itemize}
    \item<1-> extension of ASP solver \smodels
    \end{itemize}
    \bigskip
  \item<2-> \alert{\clingcon}
    \begin{itemize}
    \item<2-> extension of ASP system \clingo\ \ (viz.\ \gringo\ and \clasp)
    \item<2-> lazy approach
    \end{itemize}
  \item<3-> \alert{\aspartame}
    \begin{itemize}
    \item<3-> translational approach \ (independent of ASP system)
    \item<2-> eager approach
    \end{itemize}
    \bigskip
  \item<4->
    \alert{\aspmt},
    \alert{\dlvhex},
    \alert{\ezcsp},
    \alert{\gasp},
    \alert{\inca},
    \dots
  \end{itemize}
\end{frame}
% ----------------------------------------------------------------------
\begin{frame}{\aspartame's eager approach}
  \bigskip
  \begin{center}
  \thicklines
  \setlength{\unitlength}{1.25pt}
  \scriptsize
  \begin{picture}(280,85)(0,-10)
    \only<1>{%
    \put( 6, 20){\dashbox(24,24){\shortstack{CSP\\Instance}}}
    \put( 40, 20){\framebox(50,24){\sugar\quad}}
    \put( 80, 20){\dashbox(10,24){\it\shortstack{A\\S\\P}}}}
    \put(100, 20){\dashbox(20,24){\shortstack{ASP\\Facts}}}
    \put( 80,-10){\dashbox(40,24){\shortstack{ASP\\Encoding\only<3->{$^*$}}}}
    \put(130, 20){\framebox(50,24){\gringo}}
    \put(190, 20){\framebox(50,24){\clasp}}
    \put(250, 20){\dashbox(24,24){\shortstack{\alt<2->{CASP}{CSP}\\Solution}}}
    \only<1>{\put( 30, 32){\vector(1,0){10}}}
    \only<1>{\put( 90, 32){\vector(1,0){10}}}
    \put(120, 32){\vector(1,0){10}}
    \put(180, 32){\vector(1,0){10}}
    \put(240, 32){\vector(1,0){10}}
    \put(120, +2){\line(1,0){4}}
    \put(124, +2){\line(0,1){30}}
    \pause[2]
    \put( 80, 50){\dashbox(40,24){\shortstack{CASP\\Program}}}
    \put(120, +62){\line(1,0){4}}
    \put(124, +32){\line(0,1){30}}
  \end{picture}
\end{center}
\vfill
\begin{itemize}
\item<3-> []$^*$ based on order-encoding for CSPs
\end{itemize}
\end{frame}
% ----------------------------------------------------------------------
\begin{frame}[c]{\clingcon's lazy approach}
  \begin{center}
  \thicklines
  \setlength{\unitlength}{1.25pt}
    \begin{picture}(280,60)
    \put(  2, 20){\dashbox(33,30){\small\shortstack{CASP\\Program}}}
    \put( 50, 10){\framebox(170,50){}}
    \put( 60, 20){\framebox(60,30){\gringo\qquad}}
    \put(150, 20){\framebox(60,30){\clasp\qquad}}
    \put( 95, 25){\framebox(20,10){\small\textcolor<3>{yellow}{CSP}}}
    \put(185, 25){\framebox(20,10){\small CSP}}
    \put(235, 20){\dashbox(33,30){\small\shortstack{CASP\\Solution}}}
    \put(120, 35){\vector(1,0){30}}
    \put( 35, 35){\vector(1,0){25}}
    \put(210, 35){\vector(1,0){25}}
    \only<7->{\put(  2, 55){\dashbox(33,30){\small\shortstack{CSP\\Grammar}}}
    \put(35, +72){\line(1,0){5}}
    \put(40, +72){\line(0,-1){37}}}
    \end{picture}
  \end{center}
  \begin{minipage}[t]{0.49\linewidth}
    \begin{itemize}
    \item<2-> \clingcon\ 1\only<5->{+2}
      \begin{itemize}
      \item<3-> language extension
      \item<4-> propagation via \gecode\
      \item<5-> conflict minimization
      \end{itemize}
    \end{itemize}
  \end{minipage}
  \begin{minipage}[t]{0.49\linewidth}
    \begin{itemize}
    \item<6-> \clingcon\ 3
      \begin{itemize}
      \item<7-> language specification
      \item<8-> lazy propagation$^*$
      \end{itemize}
    \end{itemize}
  \end{minipage}
\end{frame}
% ----------------------------------------------------------------------
\begin{frame}{\clingcon\alt<2->{ instantiates \clingo}{'s approach}}
  \bigskip\smallskip
  \begin{center}
  \thicklines
  \setlength{\unitlength}{1.25pt}
    \begin{picture}(280,60)
    \put(  2, 20){\dashbox(33,30){\small\shortstack{\alt<2->{T-ASP}{CASP}\\Program}}}
    \put( 50, 10){\framebox(170,50){}}
    \put( 60, 20){\framebox(60,30){\gringo\qquad}}
    \put(150, 20){\framebox(60,30){\clasp\qquad}}
    \put( 95, 25){\framebox(20,10){\small\alt<2->{T}{CSP}}}
    \put(185, 25){\framebox(20,10){\small\alt<2->{T}{CSP}}}
    \put(235, 20){\dashbox(33,30){\small\shortstack{\alt<2->{T-}{C}ASP\\Solution}}}
    \put(120, 35){\vector(1,0){30}}
    \put( 35, 35){\vector(1,0){25}}
    \put(210, 35){\vector(1,0){25}}
    \put(  2, 55){\dashbox(33,30){\small\shortstack{\alt<2->{Theory T}{CSP}\\Grammar}}}
    \put(35, +72){\line(1,0){5}}
    \put(40, +72){\line(0,-1){37}}
  \end{picture}
  \end{center}
\end{frame}
% ----------------------------------------------------------------------
